\documentclass{inetese}
\usepackage{isolatin1}       %para os acentos iso
\usepackage{graphicx}        %incluir figuras encapsulated postscript
\usepackage{babel}

\ineTitulo{JMinimizer: Um Compactador de aplica��es java.}

\ineAutor{Thiago Le�o Moreira}
\ineOrientador{Prof. Dr. Ant�nio Augusto Medeiros Fr�hlich}

\ineAreaConcentracao{Ferramentas de desenvolvimento}
\ineTipoTese{disserta��o}

\ineGrau{Bacharel}
\ineMes{Novembro}
\ineAno{2004}

\ineCoordenadorCurso{Prof. Dr. Jos� Mazzucco Junior}
\ineMembroBancaA{Prof. Dr. Luiz Carlos Zancanella}
\ineMembroBancaB{M. Sc. Tiago Stein D\'Agostini}

\begin{document}

\inePaginaDeRosto
\inePaginaDeAprovacao

\begin{ineEpigrafe}
``A verdadeira fun��o do homem � viver \\
  e n�o apenas existir.''\\
- Jack London
\end{ineEpigrafe}

\begin{ineOferecimento}
  ``A mim, pelas noite n�o dorminadas...''
\end{ineOferecimento}

%\begin{ineAgradecimentos}
%  Agradecimentos...
%\end{ineAgradecimentos}

\begin{ineResumo}
O crescente aumento da popularidade da linguagem de programa��o Java levou tamb�m a um aumento, no desenvolvimento de frameworks e bibliotecas utilit�rias, que facilitam o desenvolvimento de aplica��es. No entanto estes frameworks e bibliotecas aumentam o tamanho da aplica��o final, podendo at� ultrapassar em muitas vezes o tamanho da aplica��o que efetivamente resolve o problema de neg�cio. Frameworks e bibliotecas geralmente s�o desenvolvidos para resolver ou tratar de problemas para uma grande quantidade de situa��es. No entanto aplica��es s�o criadas para resolver um problema espec�fico, sendo assim essas aplica��es n�o utilizam todos os recursos que um framework ou uma biblioteca oferecem. � nessas funcionalidades in�teis para a aplica��o espec�fica que o JMinimizer far� e/ou remover� as mesmas.

  \paragraph{Keywords:} Java, J2ME, celular, PDA's.
\end{ineResumo}
                                                                                                                            
\begin{abstract}
  Abstract
  \paragraph{Keywords:} Java, J2ME, celular, PDA's.
\end{abstract}

% Insere o Sum�rio
\tableofcontents    \clearpage
                                                                                                                            
% Insere a lista de Figuras e de Tabelas
\listoffigures \clearpage
%\listoftables \clearpage

\chapter{Introdu��o}

Este trabalho consiste do estudo e desenvolvimento de uma aplica��o capaz de analisar uma outra aplica��o Java\cite{java} e apartir dessa analise transformar a aplica��o, de forma que seu comportamante n�o se modifique, tentando reduzir o seu tamanho original.

\section{Motiva��o}
A popularidade da linguagem de programa��o Java\cite{java}, principalmente para pequenos dispositivos\cite{j2me} %COLOCAR AQUI INFO SOBRE A QTD DE CELULARES %
 (celulares, PDA's \footnote{Personal Digital Assistant}, smart phones, etc), resultou num aumento significante de bibliotecas utilit�rias e frameworks que facilitam e aumentam a produtividade no desenvolvimento de aplica��es para esta linguagem. Tais bibliotecas e frameworks s�o utilizados como infra-estrutura na solu��o de problemas espec�ficos.
Bibliotecas para logging, manipula��o de documentos XML, constru��o de interfaces gr�ficas, frameworks para desenvolvimento WEB, para persist�ncia de dados, etc\ldots s�o algumas aplica��es que estas bibliotecas de classes possuem. 
Sites como http://ws.apache.org, http://java.net, http://jakarta.apache.org, http://www.sf.net s�o web sites especializados em abrigar projetos de frameworks e bibliotecas para a linguagem Java\cite{java}, neles s�o disponibilizados dezenas e at� centenas de pequenos e grandes projetos destinados a facilitar o desenvolvimento de aplica��es \cite{java}, sendo ela para qualquer uma das tr�s plataformas: J2ME\footnote{Java 2 Micro Edition}, J2SE\footnote{Java 2 Standard Edition}, J2EE\footnote{Java 2 Enterprise Edition}. Poupando assim tempo e dinheiro de construir e depurar classes de infra-estrutura.
No entanto a utiliza��o de bibliotecas de terceiros pode acarretar num aumento do tamanho da aplica��o se estas bibliotecas n�o tiverem dispon�veis no ambiente de execu��o do aplicativo.
Em conseq��ncia do aumento do tamanho da aplica��o tamb�m aumentar� o tempo para se realizar o download (se for esta a forma de distribui��o do aplicativo) e aumentar� o espa�o necess�rio para acomodar a aplica��o no dispositivo. Este �ltimo � de suma import�ncia quando desenvolvemos aplica��es para a plataforma J2ME\footnote{Java 2 Micro Edition}, onde os dispositivos alvos podem ter somente uma pequena quantidade de espa�o para o armazenamento de aplica��es.
Isto exposto verificamos que a utiliza��o de bibliotecas de terceiros pode resolver o problema de desenvolver e depurar classes para a infra-estrutura e criar outros problemas relacionados ao armazenamento e ao tempo de obten��o da aplica��o. No entanto este segundo, parece ser de mais f�cil solu��o. Uma primeira alternativa de solu��o para o problema seria aumentar a capacidade de armazenamento do aparelho/dispositivo ou adquirir um aparelho/dispositivo similar com maior capacidade de armazenamento. Mas se n�o for poss�vel aumentar a capacidade de armazenamento, nem de trocar de aparelho/dispositivo a segunda solu��o seria tentar retirar do c�digo gerado todo o tipo de informa��o e estrutura que n�o ir� afetar a execu��o normal do aplicativo. E � nesta segunda solu��o que este trabalho de conclus�o de curso � baseado.

\section{Justificativa}

\subsection{Cient�fica}
O desenvolvimento dessa disserta��o contribuir� para a comunidade cient�fica no esclarecimento da estrutura do \textit{bytecode} Java\cite{java} e quais dessas estruturas podem ser removidas.

\subsection{Pessoal}
A grande curiosidade que sempre tive em rela��o ao \textit{bytecode} Java\cite{java}, que � meu instrumento de trabalho, e a necessidade de desenvolver um trabalho de conclus�o de curso para Universidade Federal de Santa Catarina me levaram a desenvolver essa disserta��o.

\subsection{Social}
A possibilidade de desenvolver aplica��es e compartilha-las com meus amigos, sempre foi algo que imaginei um dia fazer e a plataforma J2ME\cite{j2me}, me possibilitou realizar esse desejo. Por esse motivo desenvolvi o JMinimizer, para auxiliar desenvolvedores que como eu desejam distribuir suas aplica��es entre seus amigos.

\section{Objetivos}

\subsection{Geral}
Desenvolver uma ferramenta de desenvolvimento de aplica��es Java\cite{java} que auxilie o \textit{deployment} dessas aplica��es na maior quantidade de dispositivos quem suportam a plataforma J2ME\cite{j2me}
\subsection{Espec�ficos}
Esse projeto apresenta os seguintes objetivos espec�ficos.

\begin{itemize}
\item{Conhecimento}
\begin{itemize}
\item{Consolidar meus conhecimentos na t�cnologia Java\cite{java}.}
\item{Aprender as etapas de desenvolvimento de uma aplica��o open source.}
\end{itemize}
\item{Criar uma ferramenta capaz de diminuir o tamanho de uma aplica��o Java\cite{java}}
\end{itemize}

\include{FundamentosTeoricos}

\renewcommand\bibname{Refer�ncias Bibliogr�ficas}
\bibliographystyle{abnt} % Estilo para gerar refer�ncias em conformidade com
                         % as normas brasileiras
\bibliography{tcc}

\end{document}
